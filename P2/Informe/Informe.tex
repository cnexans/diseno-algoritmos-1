\documentclass[a4paper,10pt]{article}
\usepackage{algorithm}
\usepackage{algorithmic}
\input{sa.tex}
%\documentclass[a4paper,10pt]{scrartcl}
\usepackage[utf8]{inputenc}
\title{}
\author{}
\date{}
\pdfinfo{%
  /Title    (Diseño de Algoritmos - Proyecto 1)
  /Author   (Fabio Castro, Leopoldo Pimentel)
  /Creator  ()
  /Producer ()
  /Subject  ()
  /Keywords ()
}
\begin{document}




 
\title{\Huge Proyecto 1-Diseño de Algoritmos I}



\author{Fabio Castro 10-10132, Leopoldo Pimentel 06-40095} 



\date{05/02/2015}

\maketitle
\section{Problema 1}
\subsection{Análisis de Complejidad}
Cuando hablamos del analisis de complejidad de nuestro algoritmo, primero tenemos que ver el analisis de complejidad del brelaz
\subsection{Explicación de la solución dada}
\hspace{2cm}
Nuestra solucion al problema toma en consideracion el paper suministrado sobre el algoritmo brelaz y su modificacion para hacerlo mas optimo. 
Para implementar dicho algoritmo, usamos el lenguaje C++, con su implementacion de vectors en vez de crear una estructura para almacenar el grafo.
De esta forma, tratamos de hacer mas ligera la implementacion en memoria y usamos las herramientas que nos provee el lenguaje.

\subsection{PseudoCódigo}
\begin{algorithm}                      % enter the algorithm environment
\caption{Algoritmo de Brelaz para la coloración de grafo modificado}          % give the algorithm a caption
\label{Problema 1}                           % and a label for \ref{} commands later in the document
\begin{algorithmic}[1]                    % enter the algorithmic environment
\REQUIRE {$G$ Grafo no dirigido }
    \ENSURE $Color$ Entero con el número de colores mínimo
    \STATE bool $back$ = \FALSE
    \STATE bool $block$ = \FALSE
    
    \STATE $k$ = $w + 1$
    \STATE {marcarClicle$(G)$}
    \WHILE \TRUE
		\IF {no $back$}
			\STATE Determinar  $u_k$ \AND  $\cup(x_k)$
			\FORALL {$c$ $\in$ $\cup(x_k)$}
				\STATE {Determinar el numero de $nodos$ vecinos sin color de $x_k$ para los que no se ha usado el color $c$}		
				\STATE {Determinar el numero de $nodos$ vecinos sin color de $x_k$ bloqueados en caso de la coloracion de $x_k$ con $c$}
				\STATE { Ordenar los colores $c \in  \cup(x_k)$. Tomando en cuenta primero el numero de bloqueos y luego la cantidad de prevenciones}
								
 			\ENDFOR
 		\ELSE
 			\STATE $c$ = color de $x_k$
 			\STATE $\cup(x_k)$ = $\cup(x_k)-{c}$
 			\STATE remover $label$ de $x_k$
		\ENDIF    	 
    	\IF {$\cup(u_k) \neq \oslash $}
    		\STATE $i$ = color de orden minimal para $\cup(u_k)$
    		\IF {$i$ no es un color bloqueado}
				\STATE color del $nodo$ $x_k$ = $i$
				\STATE $k$ = $k+1$
				\IF {$k > n$}
					\STATE $q = s$
					\STATE \COMMENT {Se encontro una nueva solucion}
		    		\IF {$q=w$}
    					\STATE \COMMENT {Salir del ciclo}
    				\ENDIF
    				\STATE $k$ = Min entre los $nodo$ coloreados
    				\STATE remover todas las $labels$ de los $nodos$
    				\STATE {$\forall$ {$x_k,...,x_n$} $back$ = \TRUE}
    			\ELSE
    				\STATE $back$ = \FALSE
    			\ENDIF
    		\ELSE
    			\STATE $back$ = \TRUE
    			\STATE $block$ = \TRUE    			
    		\ENDIF
    	\ENDIF
    	\ENDWHILE
\end{algorithmic}
\end{algorithm}

\begin{algorithm}			
\caption{Continuacion}  
\begin{algorithmic}
		\WHILE \TRUE
		\STATE \COMMENT {Continuacion del ciclo pasado}
			\IF {$back$}
			\IF {$block$} 
				\FORALL {color bloqueado $c$ $\in$ $\cup(u_k)$}
					\STATE Determinar todos los $x_c$ $nodos$ bloqueados por $c$ \AND aplicar $label$ a cada uno de los $x_c$
					\STATE $block$ = \FALSE 
				\ENDFOR
			\ENDIF
			
			\STATE $label$ todos los $x_k$
			\STATE $k$ = Maximo de todos los $nodos$ con $label$
			\IF {$k$ < $w$}
				\STATE \COMMENT {Salir del Ciclo}
			\ENDIF			
		\ENDIF				   		    		    	
    \ENDWHILE
\end{algorithmic}
\end{algorithm}


\end{document}
