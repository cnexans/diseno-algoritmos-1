\documentclass[a4paper,10pt]{article}
\usepackage{algorithm}
\usepackage{algorithmic}
\input{sa.tex}
\usepackage[utf8]{inputenc}
\title{}
\author{}
\date{}
\pdfinfo{
  /Title    (Proyecto 3 - Diseño de Algoritmos I)
  /Author   (Fabio Castro, Leopoldo Pimentel)
  /Creator  (Fabio Castro, Leopoldo Pimentel)
  /Producer ()
  /Subject  ()
  /Keywords ()
}
\begin{document}
 
\title{\Huge Proyecto 3 - Diseño de Algoritmos I}
\author{Fabio Castro 10-10132, Leopoldo Pimentel 06-40095} 
\date{13/03/2015}
\maketitle


%%% Problema 1 %%%
\section{Problema ACMAKER}

  %%% Descripción del problema %%%
  \subsection{Descripción del diseño del algoritmo implementado para la solución óptima}
  \hspace{2cm}

  \subsection{Descripción de la función de optimización empleada}
  \hspace{2cm}

  \subsection{Estrategia de programación dinámica seguida}
  \hspace{2cm}

  \subsection{Análisis de complejidad en tiempo y espacio}
  \hspace{2cm}

  %%% Pseudocódigo %%%
  \begin{algorithm}                  
  \caption{Algoritmo de Brelaz para la coloración de grafo modificado}         
  \label{ACMAKER}                 
  \begin{algorithmic}[1]                 
  \REQUIRE {Variables que necesita }
      \ENSURE Variable qiue retorna
      \STATE Lineas de código
  \end{algorithmic}
  \end{algorithm}


%%% Problema 2 %%%
\section{Problema BABY}

  %%% Descripción del problema %%%
  \subsection{Descripción del diseño del algoritmo implementado para la solución óptima}
  \hspace{2cm}

  \subsection{Descripción de la función de optimización empleada}
  \hspace{2cm}

  \subsection{Estrategia de programación dinámica seguida}
  \hspace{2cm}

  \subsection{Análisis de complejidad en tiempo y espacio}
  \hspace{2cm}

  %%% Pseudocódigo %%%
  \begin{algorithm}                  
  \caption{Algoritmo de Brelaz para la coloración de grafo modificado}         
  \label{BABY}                 
  \begin{algorithmic}[1]                 
  \REQUIRE {Variables que necesita }
      \ENSURE Variable qiue retorna
      \STATE Lineas de código
  \end{algorithmic}
  \end{algorithm}



%%% Problema 3 %%%
\section{Problema BORW}

  %%% Descripción del problema %%%
  \subsection{Descripción del diseño del algoritmo implementado para la solución óptima}
  \hspace{2cm}

  \subsection{Descripción de la función de optimización empleada}
  \hspace{2cm}

  \subsection{Estrategia de programación dinámica seguida}
  \hspace{2cm}

  \subsection{Análisis de complejidad en tiempo y espacio}
  \hspace{2cm}

  %%% Pseudocódigo %%%
  \begin{algorithm}                  
  \caption{Algoritmo de Brelaz para la coloración de grafo modificado}         
  \label{BORW}                 
  \begin{algorithmic}[1]                 
  \REQUIRE {Variables que necesita }
      \ENSURE Variable qiue retorna
      \STATE Lineas de código
  \end{algorithmic}
  \end{algorithm}


%%% Problema 4 %%%
\section{Problema MAXWOODS}

  %%% Descripción del problema %%%
  \subsection{Descripción del diseño del algoritmo implementado para la solución óptima}
  \hspace{2cm}

  \subsection{Descripción de la función de optimización empleada}
  \hspace{2cm}

  \subsection{Estrategia de programación dinámica seguida}
  \hspace{2cm}

  \subsection{Análisis de complejidad en tiempo y espacio}
  \hspace{2cm}

  %%% Pseudocódigo %%%
  \begin{algorithm}                  
  \caption{Algoritmo de Brelaz para la coloración de grafo modificado}         
  \label{MAXWOODS}                 
  \begin{algorithmic}[1]                 
  \REQUIRE {Variables que necesita }
      \ENSURE Variable qiue retorna
      \STATE Lineas de código
  \end{algorithmic}
  \end{algorithm}
  
\end{document}